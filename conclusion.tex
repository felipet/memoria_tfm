\chapter{Conclusión}

En la memoria se ha tratado la contribución al desarrollo de la nueva 
arquitectura para nodos WR basada en sistemas SoC y de la mejora de la 
electrónica encargada del sistema de generación de la señal de reloj usada como 
referencia local en los dispositivos en el marco de este Trabajo Fin de Máster. 
El trabajo realizado ha sido parte de publicaciones (citadas en el contexto de 
la memoria) además de ser motivo de nuevas publicaciones que se están 
desarrollando actualmente. A parte de la explicación de la nueva organización 
de la arquitectura basada en SoC se han discutido las ventajas e inconvenientes 
del nuevo diseño basado en SoC. También se han realizado una serie de 
experimentos para comprobar las hipótesis planteadas cuyos resultados han 
permitido verificar el funcionamiento del nuevo desarrollo basado en SoC, 
reflejando además que queda aún trabajo por hacer para lograr los objetivos 
propuestos para este nueva plataforma. En cuanto a los objetivos propuestos 
para este TFM pienso que se han cumplido y prueba de ello son las publicaciones 
asociadas al trabajo realizado.

En el capítulo 5 se ha analizado la arquitectura previa para nodos WR. Además 
del análisis teórico de las limitaciones que sufre la arquitectura usada, 
basada en FPGA, se han realizado una serie de experimentos que han aportado 
datos relevantes acerca del rendimiento del protocolo WR. Se ha podido 
comprobar que la implementación actual del \gls{wrc2p} incluida en el WR-LEN 
permite sincronizar con una exactitud por debajo del ns hasta 12 nodos 
siguiendo una topología lineal. A parte, se ha buscado comprobar si existía una 
limitación al número de nodos que podían ser sincronizados en cadena, aunque 
fuese fuera del sub-nanosegundo, y se ha comprobado que el límite actual es de 
18 nodos. La línea de trabajo futuro contempla mejorar el algoritmo de control 
utilizado para la recuperación de la señal de reloj a fin de evitar esta 
limitación observada.

Parte de los resultados obtenidos en el quinto capítulo sirven de motivación 
para el desarrollo de la nueva arquitectura basada en SoC-FPGA que se expone en 
el capítulo 6. La primera fase de dicho desarrollo ha permitido migrar el 
\textit{wrc2p} a este tipo de plataformas realizando adaptaciones para soportar 
los cambios debidos a la nueva FPGA utilizada y al nuevo soporte de un SO en la 
arquitectura de nodo. La otra parte del trabajo presentado es una discusión 
sobre los cambios necesarios para realizar un mejor aprovechamiento de la nueva 
arquitectura, eliminando elementos redundantes y simplificando el diseño 
anterior que no tenía en cuenta el soporte de un microprocesador externo a la 
FPGA.

Por último se ha realizado una caracterización de las modificaciones realizadas 
en el nodo WR-ZEN con motivo de comprobar si las hipótesis de diseño en cuanto 
a prestaciones se han cumplido. Los datos obtenidos reflejan que el nuevo 
diseño se comporta correctamente como nodo WR con un rendimiento comparable al 
de otros nodos como el WR-LEN, pero se esperaba conseguir un rendimiento 
superior debido a las mejoras incorporadas en el \textit{hardware}. El análisis 
realizado ha permitido señalar que componentes deben ser mejorados para poder 
conseguir dicho incremento en el rendimiento. En concreto se necesita optimizar 
los parámetros que influyen en el algoritmo de control utilizado en WR para 
calcular la diferencia entre la señal del reloj local y la recuperada del 
dispositivo maestro. También se ha querido incluir la línea de trabajo actual 
seguida por el CERN y que se está integrando en el dispositivo actualmente. Los 
datos han reflejado que los cambios realizados permiten reducir notablemente el 
nivel de ruido del sistema, abriendo la puerta a este tipo de dispositivos para 
su aplicación en desarrollos con requisitos de temporización bastante estrictos.

Para concluir hacer un breve repaso sobre los objetivos marcados al inicio del 
trabajo. Realizar el estado de la técnica ha sido clave para entender la 
evolución de los protocolos de sincronización y los cambios que ha introducido 
el protocolo WR para poder solventar los problemas existentes en los protocolos 
precedentes. Las limitaciones han sido evaluadas de forma teórica y práctica 
mediante la ejecución de una serie de pruebas que además fueron publicadas en 
un congreso internacional. Por último los estudios relativos a la nueva 
arquitectura y del sistema de recuperación y generación de la señal fundamental 
de reloj me han permitido realizar un estudio crítico en ambos campos 
culminando en la proposición de una serie de mejoras que serán parte de mi 
línea de trabajo futura.

\section{Trabajo futuro}

A la vez que he ido analizando la problemática de cada sección he ido 
introduciendo las posibles soluciones que permiten mejorar las limitaciones 
actuales. A continuación recordaré a modo de resumen los puntos principales.

La implementación actual para la arquitectura basada en SoC no realiza un 
aprovechamiento óptimo de la nueva plataforma. En la fase inicial de la 
migración de la arquitectura de nodo de un sistema basado únicamente en FPGA a 
otro en SoC-FPGA se ha tratado de mantener la estructura del \gls{wrc2p} para 
facilitar la tarea de migración. La siguiente fase debe aprovechar el soporte 
del microprocesador ARM y de la estructura de buses existentes en el SoC para 
trasladar los módulos WR que actualmente se interconectan por medio del bus 
Wishbone al nivel superior de la jerarquía. Para ello será necesario adaptar la 
lógica de control de Wishbone a AXI. Además será necesario investigar si la 
parte del bucle de control para el cálculo del desfase se puede llevar al ARM o 
si los requisitos de determinismo impiden tal cosa. En caso de no poder migrar 
el \textit{sw} de control al ARM se podrían explorar otras alternativas como la 
implemtación en HDL a partir de algún lenguaje de descripción \textit{hw} de 
alto nivel como System-C.

Además si se pretende extender el uso de la nueva arquitectura en aplicaciones 
con requisitos de estabilidad mayores a los que los dispositivos actuales 
pueden ofrecer, es necesario investigar en la línea de nuevos esquemas de reloj 
para el sistema de generación del reloj referencia o la mejora de alguna de las 
partes como es el caso del módulo GM. El trabajo en esta línea necesita 
determinar que componentes contribuyen de manera más significativa al ruido del 
sistema para buscar soluciones al respecto.
