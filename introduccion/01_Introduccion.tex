\chapter{Introducción}

El término sincronización hace referencia a la coordinación entre los múltiples 
elementos que componen un sistema para llevar a cabo alguna acción de forma 
organizada en el tiempo. Mantener una noción común de tiempo es un factor clave 
en el que se basan muchas de las tecnologías que empleamos y de las que 
dependemos en nuestro día a día.

\incomment{El tiempo y la exactitud alcanzable como magnitud física.} Las 
fuentes más estables fabricadas hasta la fecha son los llamados relojes
atómicos. 
\incomment{Definir.} Estos son capaces de mantener un nivel de exactitud del 
orden de \gls{ns} por día, con una precisión igual a la frecuencia del
transmisor de radio que bombea el láser. \incomment{Explicar mejor y
referenciar.} \incomment{Encontrar ref a relojes ópticos de ROA}. Este tipo de 
soluciones alcanza valores de exactitud y precisión muy notables, sin embargo, 
su coste es muy elevado restringiendo el acceso a dichas fuentes de tiempo a la 
mayoría de laboratorios e instituciones que lo precisan. Los centros de 
metrología nacionales suelen ser los encargados de proveer la hora oficial del 
país en cuestión entre otros servicios ligados. Dichos centros cuentan con 
varias fuentes estables de tiempo como máseres de hidrógeno (suele
hacerse referencia al término por la contracción de su nombre en inglés: 
\acrshort{hm}) o relojes atómicos basados en patrones de haz de Cesio. 
Actualmente existe un gran interés por parte de estas entidades en el 
desarrollo de tecnologías que permitan realizar una distribución de sus 
referencias de tiempo de una manera estable y precisa.
Además de la distribución de tiempo, el campo del posicionamiento terrestre es 
otro de los grandes interesados en las tecnologías de sincronización. 
\incomment{Hablar de GNSS}. Para estos sistemas, \incomment{...}

El gran interés mostrado tanto en el ámbito académico como en el industrial por 
mejorar los protocolos de sincronización actuales conlleva que cualquier 
trabajo en la línea de la sincronización tenga gran repercusión. En concreto la 
extensión del protocolo \gls{ptp} conocida como \gls{wr} está recibiendo mucha 
atención en los últimos años gracias al buen equilibrio entre prestaciones y 
coste de la tecnología así como a la existencia de diseños de referencia 
abiertos que fomentan el desarrollo de terceros.

\section{Organización del trabajo}

La línea de este Trabajo Fin de Máster se encuadra en la dirección de la 
temática de mi futura tésis doctoral. Con la realización de este proyecto he 
podido asentar las bases de lo que será este futuro trabajo de investigación. 
Por un lado se ha detectado una posible vía de investigación en el ámbito de la 
sincronización basada en \gls{wr}, en concreto el estudio de como mejorar la 
arquitectura existente para lograr un sistema más eficiente, además de detectar 
cuales son los puntos críticos que están limitando la mejora tanto en exactitud 
como en precisión alcanzable por dicho protocolo. Por el otro, se ha podido 
asentar la base de conocimiento tanto en la propia tecnología de sincronización 
como en los aspectos relacionados: diseño basado en lógica reconfigurable, 
sistemas de recuperación de reloj, fuentes de ruido electromagnético, etc. 
Temas que serán de clave en la consecución final de la citada tésis doctoral.

Por tanto, esta memoria refleja los primeros pasos necesarios para ello. La 
estructura de la misma se divide en una serie de ocho capítulos donde se 
introducen los conceptos teóricos clave para entender el contenido de esta 
memoria, una rápida revisión al estado de la técnica en la materia de la 
sincronización para redes basadas en el intercambio de paquetes, y los aspectos 
relativos al desarrollo propiamente dicho.

El primer capítulo realiza una introducción a los conceptos relativos al mundo 
de la sincronización en redes. También se contextualiza el trabajo y se habla 
de la importancia que tiene esta temática en las tecnologías presentes y 
futuras. Por último se definen una serie de objetivos marcados para el 
desarrollo de este TFM.

En el segundo capítulo se explica la metodología de trabajo seguida y se 
detallan tanto las herramientas \textit{hardware} y \textit{software} como el 
equipo de laboratorio que se ha utilizado.

Los capítulos tercero y cuarto contienen aspectos teóricos relativos a los 
sistemas de distribución de tiempo que son necesarios para entender esta 
memoria. También se introducen brevemente algunos conceptos sobre estabilidad y 
medidas que se utilizan para caracterizar este tipo de sistemas. Además se hace 
un rápido repaso por el estado de la técnica de los protocolos de 
sincronización estándar más relevantes a lo largo de la historia de la 
computación relacionados con la tecnología \gls{wr} sobre la que se centra este 
trabajo.

Una vez se han introducido todos los aspectos teóricos, se realiza un análisis 
y caracterización de la arquitectura para nodos WR en el quinto capítulo. Los 
resultados incluidos sirven de motivación para el desarrollo de la nueva 
arquitectura propuesta en el capítulo sexto. En este capítulo se explica las 
bases de la nueva arquitectura para nodos basada en sistemas SoC-FPGA, 
explicando el desarrollo hecho hasta la fecha y realizando una serie de 
propuestas de mejora para la línea de trabajo futuro.

El capítulo séptimo pone el foco en la parte \textit{hardware} del sistema, 
analizando el sistema de recuperación y generación de reloj en la que se 
asienta la tecnología WR. Además se incluyen una serie de resultados obtenidos 
del trabajo de caracterización y prueba de mejoras que se está realizando en la 
línea.

Para cerrar se incluyen las conclusiones generales del trabajo y una breve 
discusión sobre la línea de trabajo futura con motivo de continuar lo que ha 
sido el desarrollo realizado en el marco de una futura tésis doctoral.

\section{Objetivos}

Dada la gran complejidad que tienen los sistemas WR, se trata de proyectos que 
los componen múltiples personas. El objetivo de este trabajo ha sido contribuir 
en las áreas más cercanas a lo que será mi línea de tésis para poder adquirir 
el conocimiento necesario para poder afrontar lo que será el futuro desarrollo 
del citado trabajo. Los objetivos se reparten en varias áreas de desarrollo a 
fin de poder conocer el sistema desde los tres puntos básicos: 
\textit{firmware} para FPGA, \textit{software} y \textit{hardware}.

\begin{itemize}
	\item Analizar el estado de la técnica de los principales protocolos 
	utilizados en sincronización de redes Ethernet, haciendo hincapié en el 
	nuevo protocolo denominado \gls{wr} para adquirir el conjunto de 
	conocimientos necesario para emprender un desarrollo en esta línea de 
	trabajo.
	
	\item Analizar las limitaciones actuales de los nodos \gls{wr}. Esto 
	necesita de una labor de análisis de como se contruye la arquitectura que 
	forma un nodo y la realización de una serie de pruebas que permita 
	caracterizar el comportamiento del mismo.
	
	\item Proponer mejoras a la arquitectura de referencia de nodo. Actualmente 
	se está realizando desde el grupo de distribución de tiempo de la UGR y 
	desde la empresa Seven Solutions el desarrollo de la arquitectura basada en 
	SoC-FPGA para nodos WR. Es importante entender este nuevo tipo de 
	arquitectura para poder contribuir a su desarrollo y poder realizar un 
	análisis de que aspectos se necesita mejorar en la implementación actual.
	
	\item Otro de los objetivos es comprender como trabaja la electrónica 
	encargada de la recuperación y generación de la señal de reloj fundamental 
	de WR para realizar mejoras que permitan mejorar aún más la exactitud y 
	precisión del protocolo.
\end{itemize}

\incomment{creo que son lo suficientemente ambigüos pero no quedo contento con 
ellos, acepto sugerencias.}


