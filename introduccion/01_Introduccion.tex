\chapter{Introducción}

El término sincronización hace referencia a la coordinación entre los múltiples 
elementos que componen un sistema para llevar a cabo alguna acción de forma 
simultánea en el tiempo. Mantener una noción común de tiempo es un factor clave 
en el que se basan muchas de las tecnologías que empleamos y de las que 
dependemos en nuestro día a día.

\incomment{El tiempo y la exactitud alcanzable como magnitud física.} Las 
fuentes más estables fabricadas hasta la fecha son los llamados relojes
atómicos. 
\incomment{Definir.} Estos son capaces de mantener un nivel de exactitud del 
orden de \gls{ns} por día, con una precisión igual a la frecuencia del
transmisor de radio que bombea el láser. \incomment{Explicar mejor y
referenciar.} \incomment{Encontrar ref a relojes ópticos de ROA}. Este tipo de 
soluciones alcanza valores de exactitud y precisión muy notables, sin embargo, 
su coste es muy elevado restringiendo el acceso a dichas fuentes de tiempo a la 
mayoría de laboratorios e instituciones que lo precisan. Los centros de 
metrología nacionales suelen ser los encargados de proveer la hora oficial del 
país en cuestión entre otros servicios ligados. Dichos centros cuentan con 
varias fuentes estables de tiempo como máseres de hidrógeno (suele
hacerse referencia al término por la contracción de su nombre en inglés: 
\acrshort{hm}) o relojes atómicos basados en patrones de haz de Cesio. 
Actualmente existe un gran interés por parte de estas entidades en el 
desarrollo de tecnologías que permitan realizar una distribución de sus 
referencias de tiempo de una manera estable y precisa.
Además de la distribución de tiempo, el campo del posicionamiento terrestre es 
otro de los grandes interesados en las tecnologías de sincronización. 
\incomment{Hablar de GNSS}. Para estos sistemas, 

\section{Motivación y Objetivos}

El gran interés mostrado tanto en el ámbito académico como en el industrial por 
mejorar los protocolos de sincronización actuales conlleva que cualquier 
trabajo en la línea de la sincronización tenga gran repercusión. En concreto la 
extensión del protocolo \gls{ptp} conocida como \gls{wr} está recibiendo mucha 
atención en los últimos años gracias al buen equilibrio entre prestaciones y 
coste de la tecnología así como a la disposición de diseños de referencia 
abiertos para fomentar el desarrollo de terceros.


