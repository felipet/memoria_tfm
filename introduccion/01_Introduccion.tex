\chapter{Introducción}

El término sincronización hace referencia a la coordinación entre los múltiples 
elementos que componen un sistema para llevar a cabo alguna acción de forma 
simultánea en el tiempo. Mantener una noción común de tiempo es un factor clave 
en el que se basan muchas de las tecnologías que empleamos y de las que 
dependemos en nuestro día a día.

\incomment{El tiempo y la exactitud alcanzable como magnitud física.} Las 
fuentes más estables fabricadas hasta la fecha son los llamados relojes
atómicos. 
\incomment{Definir.} Estos son capaces de mantener un nivel de exactitud del 
orden de \gls{ns} por día, con una precisión igual a la frecuencia del
transmisor de radio que bombea el láser. \incomment{Explicar mejor y
referenciar.} \incomment{Encontrar ref a relojes ópticos de ROA}. Este tipo de 
soluciones alcanza valores de exactitud y precisión muy notables, sin embargo, 
su coste es muy elevado restringiendo el acceso a dichas fuentes de tiempo a la 
mayoría de laboratorios e instituciones que lo precisan. Los centros de 
metrología nacionales suelen ser los encargados de proveer la hora oficial del 
país en cuestión entre otros servicios ligados. Dichos centros cuentan con 
varias fuentes estables de tiempo como máseres de hidrógeno (suele
hacerse referencia al término por la contracción de su nombre en inglés: 
\acrshort{hm}) o relojes atómicos basados en patrones de haz de Cesio. 
Actualmente existe un gran interés por parte de estas entidades en el 
desarrollo de tecnologías que permitan realizar una distribución de sus 
referencias de tiempo de una manera estable y precisa.
Además de la distribución de tiempo, el campo del posicionamiento terrestre es 
otro de los grandes interesados en las tecnologías de sincronización. 
\incomment{Hablar de GNSS}. Para estos sistemas, \incomment{...}



\section{Motivación y Objetivos}

El gran interés mostrado tanto en el ámbito académico como en el industrial por 
mejorar los protocolos de sincronización actuales conlleva que cualquier 
trabajo en la línea de la sincronización tenga gran repercusión. En concreto la 
extensión del protocolo \gls{ptp} conocida como \gls{wr} está recibiendo mucha 
atención en los últimos años gracias al buen equilibrio entre prestaciones y 
coste de la tecnología así como a la existencia de diseños de referencia 
abiertos que fomentan el desarrollo de terceros.

La línea de este Trabajo Fin de Máster se encuadra en la dirección de la 
temática de mi futura tésis doctoral. Con la realización de este proyecto he 
podido asentar las bases de lo que será este futuro trabajo de investigación. 
Por un lado se ha detectado una posible vía de investigación en el ámbito de la 
sincronización basada en \gls{wr}, en concreto el estudio de como mejorar la 
arquitectura existente para lograr un sistema más eficiente, además de detectar 
cuales son los puntos críticos que están limitando la mejora tanto en exactitud 
como en precisión alcanzable por dicho protocolo. Por el otro, se ha podido 
asentar la base de conocimiento tanto en la propia tecnología de sincronización 
como en los aspectos relacionados: diseño basado en lógica reconfigurable, 
sistemas de recuperación de reloj, fuentes de ruido electromagnético, etc. que 
será de vital importancia para la consecución final de la citada tésis doctoral.

Por tanto, esta memoria refleja los primeros pasos necesarios para ello. En los 
capítulos 2 y 3 se introducen los conceptos teóricos clave para entender el 
trabajo. Se incluyen nociones clave para comprender el problema de la 
sincronización y las soluciones más relevantes que se han utilizado hasta la 
fecha. También se habla brevemente de temas de electrónica y de como se mide el 
ruido en estos sistemas \incomment{no me gusta}. En el capítulo 4 se habla de 
los componentes tecnológicos utilizados: tarjetas, entornos y herramientas de 
desarrollo, etc. El quinto capítulo contiene en análisis primigenio realizado 
para detectar puntos flacos y posibles cosas a mejorar en la arquitectura de un 
dispositivo \gls{wr}. El sexto capítulo trata la propuesta de la nueva 
arquitectura basada en \gls{soc} para nodos \gls{wr} y detalla algunas de las 
mejoras realizadas hasta la fecha. Los últimos capítulos contienen ideas para 
lo que será el trabajo futuro en esta línea y la conclusiones alcanzadas 
durante la realización de este trabajo.

Dado que este trabajo se engloba dentro de una línea más extensa que llevará 
varios años, se han planteado una serie de objetivos que tiene sentido alcanzar 
durante la realización de este trabajo y que se encuadran dentro de lo que será 
la posterior realización de la tesis doctoral:

\begin{itemize}
	\item Analizar el estado de la técnica de los principales protocolos 
	utilizados en sincronización de redes Ethernet, haciendo hincapié en el 
	nuevo protocolo denominado \gls{wr}.
	
	\item Analizar las limitaciones actuales de los nodos \gls{wr}.
	
	\item Proponer mejoras a la arquitectura de referencia de nodo.
	
	\item Analizar el sistema de recuperación de reloj y proponer mejoras para 
	conseguir aumentar las prestaciones en la sincronización 
\end{itemize}


