\chapter{Metodología de trabajo y herramientas}

El trabajo realizado en el marco del TFM se ha llevado a cabo entre las 
instalaciones del edificio CITIC de la UGR y el laboratorio de sincronización 
de la empresa Seven Solutions (en el edificio CETIC de la UGR). Dada la 
naturaleza de este tipo de desarrollos se necesita una gran cantidad de 
material y herramientas que no suelen ser baratas por lo que no siempre es 
fácil disponer de ellas. En resumen se han utilizado equipos \gls{wr} como el 
WR-LEN y el WR-ZEN, equipos de medición como osciloscopios y analizadores de 
\textit{Phase Noise}, multitud de fibras y demás material para formar redes WR, 
etc.

La metodología de desarrollo ha seguido un modelo en cascada: primero se ha 
realizado una labor de análisis de la arquitectura actual en los nodos WR y de 
una implementación concreta de dicha arquitectura. Para ello se han realizado 
una serie de experimentos de caracterización que permiten extraer datos 
interesantes acerca de las limitaciones de la tecnología actual. Luego se 
introduce el nuevo desarrollo de una arquitectura basada en la familia de 
\gls{soc} Zynq-7000 de Xilinx y se explican las posiblidades futuras de mejora 
basadas en esta nueva arquitectura. Por último se realizan varios experimentos 
sobre el desarrollo inicial basado en Zynq para dejar una línea abierta sobre 
el trabajo futuro de mejora de este diseño.

Se ha trabajado con varios dispositivos WR durante el desarrollo de este 
trabajo:

\begin{itemize}
	\item WRS \incomment{decir algo y fotos}
	\item WR-ZEN \incomment{decir algo}
	\item WR-LEN \incomment{decir algo}
\end{itemize}

Además han sido necesarios múltiples dispositivos de medida de alta resolución 
debido a las características tan exigentes del procolo de sincronización WR:

\begin{itemize}
	\item Osciloscopio Tektronix DPO-7354 \incomment{decir algo}
	\item Medidor de Phase Noise Microsemi 3120A \incomment{decir algo}
	\item Frecuencímetro/Contador Keysight 53230 \incomment{decir algo}
\end{itemize}

El material típico para realizar las conexiones de fibra o controlar los 
aparatos utilizados:

\begin{itemize}
	\item SFPs \incomment{explicar algo}
	\item SFPs de cobre
	\item Fibra óptica, bobinas...
	\item Cables de RF \incomment{decir algo de sma, smc y esas mierdas}
	\item \incomment{mencionar la morralla más normal}
\end{itemize}

Además de todos los componentes físicos, también han sido necesarias múltiples 
herramientas \textit{software}:

\begin{itemize}
	\item Sistema operativo GNU/Linux, enconcreto Ubuntu. Por suerte la mayoría 
	de herramientas de desarrollo para el ámbito de WR están preparadas para 
	este tipo de SO. Se necesitan versiones concretas (la LTS de turno) ya que 
	los proyectos suelen estar preparados para compilar bajo unas condiciones 
	de paquetes y versiones específicas. Intentar usar versiones más modernas u 
	otros SO requiere un gran manejo y una gran inversión de tiempo.
	
	\item Herramientas de control de versiones, como Git, para la gestión de 
	los repositorios disponibles en la web de \gls{ohwr} \cite{website:ohwr} 
	así como para los desarrollos propios.
	
	\item Múltiples conjuntos de herramientas de compilación cruzada 
	(\textit{toolchains}) para compilar el \textit{sw} para los dispositivos WR.
	
	\item Herramientas de síntesis para \gls{hdl} de Xilinx. En concreto se ha 
	utilizado ISE para la familia Spartan 6 y Vivado para las familias Artix-7 
	y Zynq-7000.
	
	\item También son necesarias muchas pequeñas herramientas para acceso a 
	terminales serie o programación de dispositivos. Caben destacar algunas de 
	la comunidad \gls{wr} como \textit{hdl-make} que permite compilar proyectos 
	de ISE sin necesidad de utilizar la interfaz gráfica y \textit{wbgen2} que 
	se utiliza para generar módulos HDL para el bus Wishbone, en concreto dicha 
	herramienta genera 
	la lógica de control para acceso a bus y la interfaz del módulo, dejando al 
	usuario el desarrollo de la lógica que implemente la funcionalidad concreta.
\end{itemize}

Dada la complejidad de desarrollo para este tipo de sistemas, es necesaria una 
labor de aprendizaje bastante extensa a fin de manejar con soltura el conjunto 
de herramientas sw y los distintos dispositivos \textit{hw}, tanto equipos WR 
como de medición.