\chapter*{}
%\thispagestyle{empty}
%\cleardoublepage

%\thispagestyle{empty}

\input{portada/portada_2}



\cleardoublepage
\thispagestyle{empty}

\begin{center}
{\large\bfseries \myTitle}\\
\end{center}
\begin{center}
\myName\\
\end{center}

%\vspace{0.7cm}
\noindent{\textbf{Palabras clave}: White-Rabbit, Sincronización, PTP}\\

\vspace{0.7cm}
\noindent{\textbf{Resumen}}\\

Este proyecto trata sobre la aplicación de los conocimientos adquiridos en varias
asignaturas diferentes de la titulación para optimizar un algoritmo de resolución de
grandes sistemas de ecuaciones lineales, denominado Método el Gradiente Conjugado Precondicionado (PCGM).
Se aplicarán varias técnicas de optimización en aquellas zonas donde el compilador no es capaz de aprovechar
las características propias de la arquitectura objetivo de la build. También se intentará
explotar el paralelismo implícito en la aplicación que se encuentra implementada secuencialmente
con el objetivo de utilizar procesadores que aprovechan eficientemente el paralelismo como las GPUs.

La implementación utilizada se compara con otra biblioteca Fortran muy utilizada en estos ámbitos,
la library NSPCG, que no aprovecha completamente la potencia de cálculo que hay disponible.
Para mejorar el rendimiento con respecto a esta biblioteca se realizarán varios esfuerzos.
Primero, se buscará optimizar el uso de la CPU mediante el empleo de características
como las instrucciones multimedia que son capaces de realizar cálculos de forma vectorial. Por último,
mediante el uso de otra arquitectura típica de un computador como es la GPU.

\cleardoublepage


\thispagestyle{empty}


\begin{center}
{\large\bfseries Project Title: Project Subtitle}\\
\end{center}
\begin{center}
First name, Family name (student)\\
\end{center}

%\vspace{0.7cm}
\noindent{\textbf{Keywords}: Code optimization, assembly, PCGM, CUDA, SIMD, Fortran}\\

\vspace{0.7cm}
\noindent{\textbf{Abstract}}\\

This project is about the application of the acquired knowledge in many different degree`s subjects in order to optimize an algorithm of resolution of big systems of linear equations, named Preconditioned Conjugate Gradient Method (PCGM). It will apply many techniques of optimization in those areas where the compiler is not able to make the most of the own characteristics of the architecture objective of the build. In addition, it will try to utilise the implicit parallelism in the application that is implemented sequentially with the aim of using processors which profit efficiently the parallelism as the GPUs.

The used implementation is compared with another Fortran library very utilised in these areas, the NSPCG library, which does not profit completely the power of the calculation that is available. In order to improve the performance with respect to this library, it will be made several efforts. First, it will seek to optimize the CPU’s use by the employment of characteristics as the multimedia instructions that are able to make calculations in a vectorial way. Finally, through the utilization of another typical architecture of a computer as the GPU.

\chapter*{}
\thispagestyle{empty}

\noindent\rule[-1ex]{\textwidth}{2pt}\\[4.5ex]

Yo, \textbf{\myName}, alumno de la titulación \myDegree de la \textbf{Escuela Técnica Superior
de Ingenierías Informática y de Telecomunicación de la Universidad de Granada}, con DNI 15454650F, autorizo la
ubicación de la siguiente copia de mi Trabajo Fin de Grado en la biblioteca del centro para que pueda ser
consultada por las personas que lo deseen.

\vspace{6cm}

\noindent Fdo: \myName

\vspace{2cm}

\begin{flushright}
Granada a 11 de Septiembre de 2015 .
\end{flushright}


\chapter*{}
\thispagestyle{empty}

\noindent\rule[-1ex]{\textwidth}{2pt}\\[4.5ex]

D. \textbf{\myProf}, Profesor del Área de XXXX del Departamento YYYY de la Universidad de Granada.

\vspace{0.5cm}

\textbf{Informan:}

\vspace{0.5cm}

Que el presente trabajo, titulado \textit{\textbf{\myTitle}},
ha sido realizado bajo su supervisión por \textbf{\myName}, y autorizamos la defensa de dicho trabajo ante el tribunal
que corresponda.

\vspace{0.5cm}

Y para que conste, expiden y firman el presente informe en Granada a X de Septiembre de 2017 .

\vspace{1cm}

\textbf{Los directores:}

\vspace{5cm}

\noindent \textbf{\myProf}

\chapter*{Agradecimientos}
\thispagestyle{empty}

       \vspace{1cm}


Agradecer a mis tutores su dedicación y ayuda sin la que no habría sido
posible realizar este proyecto.
