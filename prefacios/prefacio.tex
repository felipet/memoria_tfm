\chapter*{}
%\thispagestyle{empty}
%\cleardoublepage

%\thispagestyle{empty}

\input{portada/portada_2}



\cleardoublepage
\thispagestyle{empty}

\begin{center}
{\large\bfseries \myTitle}\\
\end{center}
\begin{center}
\myName\\
\end{center}

%\vspace{0.7cm}
\noindent{\textbf{Palabras clave}: White-Rabbit, Sincronización, PTP, Zynq, 
SoC}\\

\vspace{0.7cm}
\noindent{\textbf{Resumen}}\\

El presente Trabajo Fin de Máster trata sobre el desarrollo de una nueva 
arquitectura basada en \textit{System on Chip} (SoC) para nodos White Rabbit 
(WR). 
Se analiza la arquitectura básica con la que tradicionalmente se han diseñado 
este tipo de dispositivos y se proponen una serie de aspectos que pueden ser 
mejorados gracias a la utilización de nuevas soluciones que integran en una  
misma plataforma microprocesador y recursos de lógica programable entre otros. 
Además se hace una revisión y caracterización de la electrónica encargada del 
sistema de reloj incorporada en los dispositivos WR actuales, y se discuten una 
serie de mejoras que llevarían a aumentar la precisión que alcanza este 
protocolo de sincronización. Finalmente se discuten las aplicaciones que se 
pueden beneficiar de los resultados presentados y se propone la línea de 
trabajo futuro.

\cleardoublepage


\thispagestyle{empty}


\begin{center}
{\large\bfseries SoC-FPGA architecture for high-accuracy synchronization 
applications}\\
\end{center}
\begin{center}
Felipe Torres González (student)\\
\end{center}

%\vspace{0.7cm}
\noindent{\textbf{Keywords}: White-Rabbit, Synchronization, PTP, Zynq, SoC, 
Timing}\\

\vspace{0.7cm}
\noindent{\textbf{Abstract}}\\

The new Zynq-based system architecture for White Rabbit (WR) nodes is explained 
in this thesis. The current WR node desing is analyzed and evaluated looking 
for an improvement on system flexibility and synchronization performance. The 
current design based on FPGA is expanded to take advantage of new system 
architecture of FPGA based SoCs. The state of art of synchronization protocols 
is also discussed and the major characteristics of the WR protocol. Moreover 
the current line of improvement of the clocking system is discussed and 
analyzed with some stability measures. Finally improvements are proposed in 
order to achieve a better precision of the synchronization accuracy. That will 
be part of the future work on the line of WR performance improvement.


\chapter*{}
\thispagestyle{empty}

\noindent\rule[-1ex]{\textwidth}{2pt}\\[4.5ex]

Yo, \textbf{\myName}, alumno de la titulación \myDegree de la \textbf{Escuela 
Técnica Superior
de Ingenierías Informática y de Telecomunicación de la Universidad de Granada}, con DNI 15454650F, autorizo la
ubicación de la siguiente copia de mi Trabajo Fin de Máster en la biblioteca 
del centro para que pueda ser consultada por las personas que lo deseen.

\vspace{6cm}

\noindent Fdo: \myName

\vspace{2cm}

\begin{flushright}
Granada a 8 de Septiembre de 2017 .
\end{flushright}


\chapter*{}
\thispagestyle{empty}

\noindent\rule[-1ex]{\textwidth}{2pt}\\[4.5ex]

D. \textbf{\myProf}, Profesor del Departamento ATC de la Universidad de Granada.

\vspace{0.5cm}

\textbf{Informan:}

\vspace{0.5cm}

Que el presente trabajo, titulado \textit{\textbf{\myTitle}},
ha sido realizado bajo su supervisión por \textbf{\myName}, y autorizamos la 
defensa de dicho trabajo ante el tribunal que corresponda.

\vspace{0.5cm}

Y para que conste, expiden y firman el presente informe en Granada a 8 de 
Septiembre de 2017 .

\vspace{1cm}

\textbf{Los directores:}

\vspace{5cm}

\noindent \textbf{\myProf}

\chapter*{Agradecimientos}
\thispagestyle{empty}

       \vspace{1cm}


Dada la complejidad que supone trabajar en la temática de la sincronización 
White Rabbit, existe una gran necesidad de ayuda por parte de compañeros de 
trabajo o de la comunidad WR. Por ello la lista de personas a las que agradecer 
su ayuda durante el tiempo que llevo trabajando en esta línea es larga y seguro 
que fácilmente olvide a alguien. Para evitarlo no haré alusión directa a 
ninguna persona, si no que agradezco la ayuda prestada tanto desde la comunidad 
WR, como por el grupo de investigación \textit{Timing Keepers} de la UGR, como 
a la proveniente de los compañeros de la empresa Seven Solutions. Espero poder 
devolver la ayuda prestada cuando esté en condiciones de poder esponder las 
dudas que reciba.

Además me gustaría agradacer la ayuda no técnica a mi familia y amigos, cuyo 
apoyo permite olvidar los malos ratos que se pasan cuando se trabaja en WR y 
todo lo que puede fallar lo hace.

Por último agradecer a mi tutor Javier Díaz Alonso el haberme introducido en 
esta línea de trabajo (aunque puede que algún día cambie de opinión al 
respecto) además de sus charlas inspiradoras que siempren levantan el ánimo.

